% \documentclass{article}
\documentclass[UTF8]{ctexart}
\usepackage{amsmath}
\usepackage{hyperref}
\usepackage{rebuttal}

\title{$title$}
% \author{$author$}

\begin{document}

\maketitle

 $body$ 

\section*{人体运动轨迹表示与坐标转换}

\subsection*{基础概念}
朝向向量是平行于地面并指向人体面朝方向的向量。通过将人体根坐标的$z$轴与世界坐标$z$轴对齐,得到的$y$轴即为朝向向量。这种方法比使用容易产生奇异性的欧拉角更稳定。

\subsection*{全局与自我中心表示}
已知每个人的无遮挡身体运动$\Theta^i$,我们需要恢复其全局轨迹$(\vec{T}^i, \vec{R}^i)$。通过全局轨迹预测器$\mathcal{T}$,将身体运动序列$\Theta = (\theta_1,\ldots,\theta_m)$转换为全局轨迹$(T, R)$:
\begin{itemize}
    \item $T = (\tau_1,\ldots,\tau_m)$:根节点平移
    \item $R = (\gamma_1,\ldots,\gamma_m)$:旋转
\end{itemize}

为处理轨迹歧义,使用CVAE构建预测器:
\begin{align}
    \Psi &= \mathcal{T}(\Theta,v) \tag{1} \\
    (T, R) &= \text{EgoToGlobal}(\Psi) \tag{2}
\end{align}
其中$v$为CVAE隐编码。

\subsection*{自我中心轨迹表示}
自我中心轨迹$\Psi$是全局轨迹$(T, R)$的等价表示,具有以下优点:
\begin{itemize}
    \item 将全局轨迹转换为相对局部差异
    \item 在朝向坐标系($y$轴对齐面朝方向)下表示旋转和平移
    \item 不受绝对$xy$平移和朝向影响
    \item 适合长轨迹预测,仅需输出逐帧局部变化
\end{itemize}

时刻$t$的自我中心轨迹分量$\psi_t = (\delta x_t, \delta y_t, z_t, \delta\phi_t, \eta_t)$计算如下:
\begin{align}
    (\delta x_t, \delta y_t) &= \text{ToHeading}(\tau^{xy}_t - \tau^{xy}_{t-1}) \tag{3} \\
    z_t &= \tau^z_t, \quad \delta\phi_t = \gamma^h_t - \gamma^h_{t-1} \tag{4} \\
    \eta_t &= \text{ToHeading}(\gamma_t) \tag{5}
\end{align}

各分量含义:
\begin{itemize}
    \item $\tau^{xy}_t$:水平平移分量
    \item $\tau^z_t$:高度分量
    \item $\gamma^h_t$:朝向角度
    \item $\eta_t$:局部旋转(采用6D表示避免不连续)
\end{itemize}

\subsection*{坐标转换}

\subsubsection*{全局转局部}
对于全局坐标$(\mathbf{T} = [\mathbf{xy}, z], \mathbf{q})$,局部转换步骤为:
\begin{align}
    \mathbf{q}' &= \mathbf{q} \otimes \mathbf{q}_{\text{base}}^* \\
    \theta &= \text{getHeading}(\mathbf{q}') \\
    \mathbf{q}_h &= \text{getHeadingQuaternion}(\mathbf{q}') \\
    \mathbf{q}_l &= \text{deheadingQuaternion}(\mathbf{q}', \mathbf{q}_h) \\
    \mathbf{R}_l &= \text{quaternionToRot6D}(\mathbf{q}_l) \\
    \Delta\mathbf{xy}_{\text{local}} &= \text{rot2D}(\mathbf{xy}_{t+1} - \mathbf{xy}_t, -\theta_t)
\end{align}

\subsubsection*{局部转全局}
对于局部坐标$\mathbf{\Psi} = [\Delta\mathbf{xy}_{\text{local}}, z, \mathbf{R}_l, \mathbf{v}_{\theta}]$,全局转换步骤为:
\begin{align}
    \theta_{\text{global}} &= \sum_{t=0}^T \text{vectorToHeading}(\mathbf{v}_{\theta}) \\
    \mathbf{xy} &= \sum_{t=0}^T \text{rot2D}(\Delta\mathbf{xy}_{\text{local}}, \theta_{\text{global}}) \\
    \mathbf{q} &= \text{headingToQuaternion}(\theta_{\text{global}}) \otimes \text{rot6DToQuaternion}(\mathbf{R}_l) \otimes \mathbf{q}_{\text{base}}
\end{align}

\subsection*{实现要点}
\begin{itemize}
    \item 初始值$(\delta x_0, \delta y_0)$和$\delta\phi_0$存储起始位置和朝向
    \item 训练使用真实初始值,推理允许任意起始条件
    \item 通过视频证据匹配修正轨迹漂移
    \item 保持四元数单位化
    \item 使用6D旋转表示避免旋转序列不连续
\end{itemize}

\end{document}

