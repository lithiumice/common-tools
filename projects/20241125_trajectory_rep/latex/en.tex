% \documentclass{article}
% \usepackage{markdown}
% \begin{document}
% \begin{markdown}

% \documentclass{article}
\documentclass[UTF8]{ctexart}
\usepackage{amsmath}
\usepackage{hyperref}
\usepackage{rebuttal}

\title{$title$}
% \author{$author$}

\begin{document}

\maketitle


\section*{Human Motion Trajectory Representation and Transformations}

\subsection*{Preliminaries}
The heading vector of a person points towards where the person is facing and is parallel to the ground. We obtain this heading vector by aligning the $z$-axis of the person's root coordinate with the world $z$-axis and using the resulting $y$-axis as the heading vector. This approach is more stable than using Euler angles, which can suffer from singularities.

\subsection*{Global and Egocentric Representations}
Given an occlusion-free body motion $\Theta^i$ for each person, we aim to recover their global trajectory $(\vec{T}^i, \vec{R}^i)$. This is achieved through a global trajectory predictor $\mathcal{T}$ that processes general body motion $\Theta = (\theta_1,\ldots,\theta_m)$ to output global trajectory $(T, R)$, where:
\begin{itemize}
    \item $T = (\tau_1,\ldots,\tau_m)$ represents root translations
    \item $R = (\gamma_1,\ldots,\gamma_m)$ represents rotations
\end{itemize}

To handle trajectory ambiguity, we formulate the predictor using a CVAE:
\begin{align}
    \Psi &= \mathcal{T}(\Theta,v) \tag{1} \\
    (T, R) &= \text{EgoToGlobal}(\Psi) \tag{2}
\end{align}
where $v$ is the CVAE latent code.

\subsection*{Egocentric Trajectory Representation}
The egocentric trajectory $\Psi$ provides an alternative representation of the global trajectory $(T, R)$ with several advantages:
\begin{itemize}
    \item Converts global trajectory into relative local differences
    \item Represents rotations and translations in heading coordinates ($y$-axis aligned with facing direction)
    \item Invariant to absolute $xy$ translation and heading
    \item More suitable for long trajectory prediction due to local frame-by-frame changes
\end{itemize}

For each time step $t$, the egocentric trajectory component $\psi_t = (\delta x_t, \delta y_t, z_t, \delta\phi_t, \eta_t)$ is computed as:
\begin{align}
    (\delta x_t, \delta y_t) &= \text{ToHeading}(\tau^{xy}_t - \tau^{xy}_{t-1}) \tag{3} \\
    z_t &= \tau^z_t, \quad \delta\phi_t = \gamma^h_t - \gamma^h_{t-1} \tag{4} \\
    \eta_t &= \text{ToHeading}(\gamma_t) \tag{5}
\end{align}

where:
\begin{itemize}
    \item $\tau^{xy}_t$ is the horizontal translation component
    \item $\tau^z_t$ is the height component
    \item $\gamma^h_t$ is the heading angle
    \item $\eta_t$ is the local rotation (using 6D representation to avoid discontinuity)
\end{itemize}

\subsection*{Coordinate Transformations}

\subsubsection*{Global to Local Transformation}
Given global coordinates $(\mathbf{T} = [\mathbf{xy}, z], \mathbf{q})$, the local transformation proceeds as:
\begin{align}
    \mathbf{q}' &= \mathbf{q} \otimes \mathbf{q}_{\text{base}}^* \\
    \theta &= \text{getHeading}(\mathbf{q}') \\
    \mathbf{q}_h &= \text{getHeadingQuaternion}(\mathbf{q}') \\
    \mathbf{q}_l &= \text{deheadingQuaternion}(\mathbf{q}', \mathbf{q}_h) \\
    \mathbf{R}_l &= \text{quaternionToRot6D}(\mathbf{q}_l) \\
    \Delta\mathbf{xy}_{\text{local}} &= \text{rot2D}(\mathbf{xy}_{t+1} - \mathbf{xy}_t, -\theta_t)
\end{align}

\subsubsection*{Local to Global Transformation}
Given local coordinates $\mathbf{\Psi} = [\Delta\mathbf{xy}_{\text{local}}, z, \mathbf{R}_l, \mathbf{v}_{\theta}]$, the global transformation is:
\begin{align}
    \theta_{\text{global}} &= \sum_{t=0}^T \text{vectorToHeading}(\mathbf{v}_{\theta}) \\
    \mathbf{xy} &= \sum_{t=0}^T \text{rot2D}(\Delta\mathbf{xy}_{\text{local}}, \theta_{\text{global}}) \\
    \mathbf{q} &= \text{headingToQuaternion}(\theta_{\text{global}}) \otimes \text{rot6DToQuaternion}(\mathbf{R}_l) \otimes \mathbf{q}_{\text{base}}
\end{align}

\subsection*{Implementation Notes}
\begin{itemize}
    \item Initial values $(\delta x_0, \delta y_0)$ and $\delta\phi_0$ store starting position and heading
    \item Training uses ground truth initial values; inference allows arbitrary starting conditions
    \item Trajectory drift correction is handled through video evidence matching
    \item All quaternion operations maintain unit normalization
    \item The 6D rotation representation prevents discontinuities in rotation sequences
\end{itemize}


% \end{markdown}
% \end{document}

\end{document}
